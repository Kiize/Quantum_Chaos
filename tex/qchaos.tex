\documentclass[11pt]{article}
\usepackage{preamble}
\usepackage{import}
\usepackage{csquotes}
\usepackage{makeidx}
\usepackage{bookmark}
\usepackage[backend=biber, sorting=none, style=phys]{biblatex}

%\addbibresource{refs.bib}

\title{Quantum chaos}
\author{V}
\date{2025}

\begin{document}
\maketitle
\section{Introduction}
We want to understand quantum chaos, where chaos manifest itself in the statistical properties of our system.
To do this, let us consider a quantum system whose classical counterpart is chaotic: the quantum billiard. 

\section{Quantum rectangular billiard}
Let us start from the simplest geometry: the rectangle, that is we want to solve the Helmholtz equation 
\begin{equation}
    -\nabla^2 \psi = E_n \psi, \quad \nabla = \left(\frac{\partial^2}{\partial x^2} + \frac{\partial^2}{\partial y^2}\right),
\end{equation}
on a rectangular domain $\Omega$ of size $L_x \times L_y$ and Dirichlet boundary conditions $\psi|_{\partial \Omega} = 0$. 
We can then compare the numerical eigenvalues found with the analytical solution Fig.~\ref{fig:energy_comparison}
\begin{equation}
    E_{n,m} = \pi^2(\frac{n^2}{L_x^2} + \frac{m^2}{L_y^2}), \quad n,m \in \mathbb{N}.
\end{equation}
\begin{figure}[h]
    \centering
    \includegraphics[width=0.5\textwidth]{../figs/rect_energy_comparison_100.png}
    \caption{Comparison between numerical and analytical eigenvalues for the rectangular billiard. The red line is $y=x$.}
    \label{fig:energy_comparison}
\end{figure}
To perform the numerical computation, we first discretize the domain and then we build the Hamiltonian matrix corresponding to the Laplacian operator using finite differences: we start from the $1D$ Laplacian with step $h$
\begin{equation}
    u^{''}_i = \frac{u_{i+1} - 2u_i + u_{i - 1}}{h^2}
\end{equation} 
from which we can build the tridiagonal matrix
\begin{equation}
    A = \frac{1}{h^2} \begin{pmatrix}
        -2 & 1 & 0 & \cdots & 0 \\
        1 & -2 & 1 & \cdots & 0 \\
        0 & 1 & -2 & \cdots & 0 \\
        \vdots & \vdots & \vdots & \ddots & \vdots \\
        0 & 0 & 0 & \cdots & -2
    \end{pmatrix}.
\end{equation}
Then we obtain the $2D$ Laplacian using the Kronecker product
\begin{equation}
    \nabla^2= A_x \otimes I_y + I_x \otimes A_y,
\end{equation}
where $I$ is the identity matrix of appropriate size. Finally, the Hamiltonian is given by $H = -\nabla^2$.


\end{document}